We have discussed arbitrary variations of wave function $|\Psi\rangle$ that don't affect parameters of hamiltonian $\hat{H}$.
To write Dirac--Frenkel variational principle in the most general form, we need to consider variations $|\delta\Psi\rangle$ of the following form:
$$|\delta\Psi\rangle=\left|\frac{\partial\Psi}{\partial\varepsilon}\right\rangle\delta\varepsilon,\ \hat{H}=\hat{H}(\varepsilon)$$
To preserve the form of DFVP for that kind of variation, we need to introduce one more condition,
 that should be met by approximate wave function $|\Psi\rangle$.
Let us consider the following equation:
$$\left\langle\Phi_{ex}(t)\left|\frac{\partial\hat{H}}{\partial\varepsilon}\right|\Phi_{ex}(t)\right\rangle = %
  i\hbar\frac{\partial}{\partial t}\left\langle\Phi_{ex}(t)\left|\frac{\partial\Phi_{ex}(t)}{\partial\varepsilon}\right.\right\rangle$$
where $|\Phi_{ex}(t)\rangle$ --- exact solution of time dependent Shr\"{o}dinger equation.
The equation is a statement of time dependant Hellmann--Feynman theorem (tdHFT). 
To prove it, we will consider the following matrix element:
$$\frac{\partial}{\partial\varepsilon}\langle\Phi_{ex}(t)|\hat{H}|\Phi_{ex}(t)\rangle=%
				      \left\langle\frac{\partial\Phi_{ex}(t)}{\partial\varepsilon}\left|\hat{H}\right|\Phi_{ex}(t)\right\rangle+%
				      \left\langle\Phi_{ex}(t)\left|\frac{\partial\hat{H}}{\partial\varepsilon}\right|\Phi_{ex}(t)\right\rangle+$$
$$ 				     -\left\langle\Phi_{ex}(t)\left|\hat{H}\right|\frac{\partial\Phi_{ex}(t)}{\partial\varepsilon}\right\rangle=$$
$$=i\hbar\left\langle\frac{\partial\Phi_{ex}(t)}{\partial\varepsilon}\left|\frac{\partial\Phi_{ex}(t)}{\partial t}\right.\right\rangle-%
   i\hbar\left\langle\frac{\partial\Phi_{ex}(t)}{\partial t}\left|\frac{\partial\Phi_{ex}(t)}{\partial\varepsilon}\right.\right\rangle+
   \left\langle\Phi_{ex}(t)\left|\frac{\partial\hat{H}}{\partial\varepsilon}\right|\Phi_{ex}(t)\right\rangle$$
$$\left\langle\Phi_{ex}(t)\left|\frac{\partial\hat{H}}{\partial\varepsilon}\right|\Phi_{ex}(t)\right\rangle=$$
$$=i\hbar\frac{\partial}{\partial\varepsilon}\left\langle\Phi_{ex}(t)\left|\frac{\partial\Phi_{ex}(t)}{\partial t}\right.\right\rangle+%
   i\hbar\left\langle\frac{\partial\Phi_{ex}(t)}{\partial t}\left|\frac{\partial\Phi_{ex}(t)}{\partial\varepsilon}\right.\right\rangle-%
   i\hbar\left\langle\frac{\partial\Phi_{ex}(t)}{\partial\varepsilon}\left|\frac{\partial\Phi_{ex}(t)}{\partial t}\right.\right\rangle=$$
$$=i\hbar\left\langle\Phi_{ex}(t)\left|\frac{\partial^2\Phi_{ex}(t)}{\partial\varepsilon\partial t}\right.\right\rangle+%
   i\hbar\left\langle\frac{\partial\Phi_{ex}(t)}{\partial t}\left|\frac{\partial\Phi_{ex}(t)}{\partial\varepsilon}\right.\right\rangle=$$
$$=i\hbar\frac{\partial}{\partial t}\left\langle\Phi_{ex}(t)\left|\frac{\partial\Phi_{ex}(t)}{\partial\varepsilon}\right.\right\rangle$$
Now we need to consider $\delta W$ in terms of new type of variations:
$$\delta W = \frac{\delta A - W\delta B}{B}$$
As we remember, $\delta B = 0$. Thus, to ensure $\delta W = 0$, we need $\delta A = 0$:
$$\delta\left\langle\Psi\left|\hat{H}-i\hbar\frac{\partial}{\partial t}\right|\Psi\right\rangle=%
	\left\langle\delta\Psi\left|\hat{H}-i\hbar\frac{\partial}{\partial t}\right|\Psi\right\rangle + %
	\left\langle\delta\Psi\left|\hat{H}-i\hbar\frac{\partial}{\partial t}\right|\Psi\right\rangle^*-$$
$$      -i\hbar\frac{\partial}{\partial t}\left\langle\Psi\left|\frac{\partial\Psi}{\partial\varepsilon}\right.\right\rangle\delta\varepsilon+%
	\left\langle\Psi\left|\frac{\partial\hat{H}}{\partial\varepsilon}\right|\Psi\right\rangle\delta\varepsilon
	$$
We need to make our function $|\Psi\rangle$ to behave in such a way, 
that tdHFT will be valid. 
Then we will obtain DFVP.
