For basis functions $|\phi_k\rangle$ we can choose frozen--width Gaussian wave packets (fwGWP):
$$|g_k(\vec{\xi}_k,\vec{\eta}_k)\rangle=\text{exp}\left(\left[\sum_{\alpha=x,y,z}-\frac{1}{2}\omega r_{\alpha}^2+\xi_{k\alpha}r_{\alpha}+\eta_{k\alpha}\right]\right)$$
where $\omega$ --- time independent parameter, that can be set equal to width of zeroth eigenfunction of harmonic oscillator.
We can transform parameters $(\vec{\xi}_k,\vec{\eta}_k)$ to $(\vec{q}_k,\vec{p}_k)$: 
$$\xi_{k\alpha} = \omega q_{k\alpha}+i p_{k\alpha}$$
$$\eta_{k\alpha} = \frac{1}{4}\left(\ln\left[{\frac{\omega}{\pi}}\right]-2\omega q_{k\alpha}^2\right)-iq_{k\alpha}p_{k\alpha}$$
where $\vec{q}_k$ --- center of gaussian wave packets, $\vec{p}_k$ --- momentum of wave packet. 
We can assume, that wave packet, centered in phase space at $(\vec{q}_k,\vec{p}_k)$, represents particle at the same point of phase space.
In this representation gaussian wave packet looks like:
$$|g_k(\vec{q}_{k},\vec{p}_{k})\rangle = \left(\frac{\pi}{\omega}\right)^{3/4} \text{exp}\left(\sum_{\alpha=x,y,z}\left[-\frac{1}{2}\omega(r_{\alpha}-q_{k\alpha})^2+%
									 ip_{k\alpha}(r_{\alpha}-q_{k\alpha})\right]\right)=g_{kx}g_{ky}g_{kz}$$

The trick of GWP basis is as follows: we generate basis, using grid in phase space, 
then switch from parameters $(\vec{q}_k,\vec{p}_k)$ to $(\vec{\xi}_k,\vec{\eta}_k)$, 
where we can easily calculate matrix elements for equations of motion (see below).
But then we understand, that actually we do not need to propagate both $\xi_k$ and $\eta_k$,
rather we can calculate only $\dot{\xi}_k$, switch back to $\dot{q}_k=\mathit{Re}(\dot{\xi}_k)/\omega$ 
and $\dot{p}_k=\mathit{Im}(\dot{\xi}_k)$ and calculate $q_k(t+\Delta t)$ and $p_k(t+\Delta t)$.

The matrix elements, that we need to use in equations of motions:
$$\mathbbm{S}_{mk,\alpha}=\text{exp}\left(\frac{(\xi_{m\alpha}^*+\xi_{k\alpha})^2}{4\omega}+\eta_{m\alpha}^*+\eta_{k\alpha} \right)$$
$$\mathbbm{S}_{mk} = \text{exp}\left(\sum_{\alpha} \frac{(\xi_{m\alpha}^*+\xi_{k\alpha})^2}{4\omega}+\eta_{m\alpha}^*+\eta_{k\alpha}\right)=%
  \mathbbm{S}_{mk,x}\mathbbm{S}_{mk,y}\mathbbm{S}_{mk,z}$$
$$\mathbbm{S}_{mk}^{(\alpha0)}=\left.\left\langle\frac{\partial g_m}{\partial\xi_{m\alpha}}\right|g_k\right\rangle=%
  \langle g_{m\alpha}|r_{\alpha}|g_{k\alpha}\rangle\left(\underset{\beta\neq\alpha}{\Pi}\mathbbm{S}_{mk,\beta}\right)=%
  \frac{\left(\xi_{m\alpha}^*+\xi_{k\alpha}\right)}{2\omega}\mathbbm{S}_{mk,x}\mathbbm{S}_{mk,y}\mathbbm{S}_{mk,z}$$
$$\mathbbm{S}_{mk}^{(0\alpha)}=\left\langle g_m\left|\frac{\partial g_k}{\partial\xi_{k\alpha}}\right.\right\rangle=%
  \langle g_{m\alpha}|r_{\alpha}|g_{k\alpha}\rangle\left(\underset{\beta\neq\alpha}{\Pi}\mathbbm{S}_{mk,\beta}\right)=%
  \frac{\left(\xi_{m\alpha}^*+\xi_{k\alpha}\right)}{2\omega}\mathbbm{S}_{mk,x}\mathbbm{S}_{mk,y}\mathbbm{S}_{mk,z}$$
$$\mathbbm{S}_{mk}^{(\alpha\beta)}=\left\langle\frac{\partial g_m}{\partial\xi_{m\alpha}}\left|\frac{\partial g_k}{\partial\xi_{k\beta}}\right.\right\rangle=%
  \langle g_{m\alpha}|r_{\alpha}|g_{k\alpha}\rangle\langle g_{m\beta}|r_{\beta}|g_{k\beta}\rangle\mathbbm{S}_{mk,\gamma}=$$
$$=\frac{\left(\xi_{m\alpha}^*+\xi_{k\alpha}\right)\left(\xi_{m\beta}^*+\xi_{k\beta}\right)}{4\omega^2}\mathbbm{S}_{mk,x}\mathbbm{S}_{mk,y}\mathbbm{S}_{mk,z}$$
$$\mathbbm{S}_{mk}^{(\alpha\alpha)}=\langle g_{k\alpha}|r_{\alpha}^2|g_{m\alpha}\rangle\left(\underset{\beta\neq\alpha}{\Pi}\mathbbm{S}_{mk,\beta}\right)=%
  \left(\frac{\left(\xi_{m\alpha}^*+\xi_{k\alpha}\right)^2}{4\omega^2}+\frac{1}{2\omega}\right)\mathbbm{S}_{mk,x}\mathbbm{S}_{mk,y}\mathbbm{S}_{mk,z}$$
$$\mathbbm{H}_{mk}=\langle g_m|\hat{H}|g_k\rangle=\langle g_m|\hat{T}_N|g_k\rangle+\langle g_m|V(\vec{r})|g_k\rangle=$$
$$=\frac{\mathbbm{S}_{mk,x}\mathbbm{S}_{mk,y}\mathbbm{S}_{mk,z}}{M}%
		       \sum_{\alpha}\left( \frac{1}{2}\left(\omega-\xi_{k\alpha}^2\right)+%
		       \xi_{k\alpha}\frac{\left(\xi_{m\alpha}^*+\xi_{k\alpha}\right)}{2}-%
		       \left(\frac{\left(\xi_{m\alpha}^*+\xi_{k\alpha}\right)^2}{8}+\frac{\omega}{4}\right)\right)+$$
$$+\langle g_m|V(\vec{r})|g_k\rangle=$$
$$=\frac{\mathbbm{S}_{mk,x}\mathbbm{S}_{mk,y}\mathbbm{S}_{mk,z}}{M}%
   \sum_{\alpha}\left(\frac{\omega+2\xi_{m\alpha}^*\xi_{k\alpha}}{4}-\frac{(\xi_{m\alpha}^*+\xi_{k\alpha})^2}{8} \right)+\langle g_m|V(\vec{r})|g_k\rangle=$$
$$=\frac{\mathbbm{S}_{mk,x}\mathbbm{S}_{mk,y}\mathbbm{S}_{mk,z}}{M}%
   \sum_{\alpha}\left(\frac{2\omega-(\xi_{m\alpha}^*-\xi_{k\alpha})^2}{8} \right)+\langle g_m|V(\vec{r})|g_k\rangle$$
$$\mathbbm{H}_{mk}^{(\alpha0)}=\left.\left.\left\langle\frac{\partial g_m}{\partial\xi_{m\alpha}}\right|\hat{H}\right|g_k\right\rangle=%
  -\frac{1}{M}\sum_{\beta}\left\langle g_m\left|r_{\alpha}\frac{\partial^2}{\partial\beta^2}\right|g_k\right\rangle+\langle g_m|V(\vec{r})|g_k\rangle=$$
$$=-\frac{1}{M}\left\langle g_{m\alpha}\left|r_{\alpha}\frac{\partial^2}{\partial\alpha^2}\right|g_{k\alpha}\right\rangle\left(\underset{\beta\neq\alpha}{\Pi}\mathbbm{S}_{mk,\beta}\right)+%
   \langle g_m|V(\vec{r})|g_k\rangle+$$
$$ +\frac{\mathbbm{S}_{mk,x}\mathbbm{S}_{mk,y}\mathbbm{S}_{mk,z}}{M}\frac{\xi_{m\alpha}^*+\xi_{k\alpha}}{2\omega}%
    \sum_{\beta\neq\alpha}\left(\frac{2\omega-(\xi_{m\beta}^*-\xi_{k\beta})^2}{8}\right)=$$
$$=\frac{\mathbbm{S}_{mk,x}\mathbbm{S}_{mk,y}\mathbbm{S}_{mk,z}}{M}\frac{\xi_{m\alpha}^*+\xi_{k\alpha}}{2\omega}%
    \sum_{\beta}\left(\frac{2\omega-(\xi_{m\beta}^*-\xi_{k\beta})^2}{8}\right)+\langle g_m|V(\vec{r})|g_k\rangle$$
For all matrix elements, except for the mean value of electron potential, we have obtained explicit equations.
For mean value of potential energy we need potential energy surface.
We can use harmonic decomposition (analytical equations) 
or numerical techniques (Gaussian quadratures) to calculate $\langle g_m|V(\vec{r})|g_k\rangle$.

