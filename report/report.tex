\documentclass[10pt,pdf,hyperref={unicode}]{beamer}
% Стандартные формульные пакеты
\usepackage{float,amsmath,esint,amsfonts,wrapfig,bbm}
\usepackage{multirow}
% Русский текст в формулах
\usepackage{mathtext}
% Подключение русского языка
\usepackage[T2A]{fontenc}
\usepackage[english]{babel}
\usepackage[utf8]{inputenc}
% Рисунки
\usepackage{graphicx,caption,subcaption}
%Theme
\usetheme{Boadilla}
%\usecolortheme{dove}

\usepackage{ragged2e}

\setbeamertemplate{frametitle}[default][center]

\addtobeamertemplate{navigation symbols}{}{%
    \usebeamerfont{footline}%
    \usebeamercolor[fg]{footline}%
    \hspace{1em}%
    \large\insertframenumber/\inserttotalframenumber
}

\newcommand{\mytitle}[1]{\color{red}{\textbf{ #1 } }}
\newcommand{\myvec}[1]{\left| #1 \right\rangle}
\newcommand{\mymean}[3]{\left\langle #1 \left| #2 \right| #3 \right\rangle }
\newcommand{\myel}[2]{\langle #1 | #2 \rangle }
\newcommand{\myelr}[2]{\left\langle #1 \left| #2 \right.\right\rangle }
\newcommand{\myell}[2]{\left.\left\langle #1 \right| #2 \right\rangle }
\newcommand{\myexp}[1]{\text{exp}\left(#1\right)}

\begin{document}
\begin{frame}{\center\mytitle{\Large Time-dependent variational principles: \\ application for quantum dynamics}}
\begin{table}[]
\flushright
\begin{tabular}{r}
\large Student of 515 group\\
\large Paramonov Nikita\\
\end{tabular}
\end{table}
\vfill
\center
\today
\end{frame}

\begin{frame}{\mytitle{ How do we calculate quantum dynamics? } }

\begin{block}{Time-dependent Shr\"odinger equation:}
$$i\hbar\frac{\partial\Phi}{\partial t}=\hat{H}\Phi$$
Let's solve it formally:
$$\Phi(t) = \Phi(0)\myexp{-\frac{i}{\hbar}\hat{H}t}\Rightarrow\frac{\partial}{\partial t}\myel{\Phi(t)}{\Phi(t)}=0$$
We can use evolution operator to calculate dynamics of wave function --- \textbf{Split operator technique}.
\end{block}
\begin{columns}
\begin{column}{0.45\textwidth}
\begin{block}{Pros}
\begin{itemize}
\item accurate ($\Delta t^3$)
\end{itemize}
\end{block}
\end{column}

\begin{column}{0.45\textwidth}
\begin{block}{Cons}
\begin{itemize}
\item slow
\end{itemize}
\end{block}
\end{column}
\end{columns}

\center{\textbf{Is there any other way to estimate accurate quantum dynamics}?}

\end{frame}

\begin{frame}{\mytitle{ Time--dependent variational principleS(!) } }

\begin{block}{Time-dependent variational principle (TDVP):}
$$\delta S = \delta\left(\int_{0}^{t} L(\Phi,\Phi^*)\,d\tilde{t}\right)=0,\ %
L(\Phi,\Phi^*) = \frac{ \mymean{\Phi} %
				 { \hat{H}-i\hbar\frac{\partial}{\partial t} }  %
				 { \Phi } }{ \myel{\Phi}{\Phi} }$$
\end{block}

\begin{block}{McLachlan's variational principle:}
$$\delta\left(\frac{ \mymean{\Phi} %
	 	{ \hat{H}-i\hbar\frac{\partial}{\partial t} }  %
		{ \Phi } }{ \myel{\Phi}{\Phi} } \right) = 0$$
\end{block}

\begin{block}{Dirac--Frenkel variational principle:}
$$\mymean{\delta\Phi} { \hat{H}-i\hbar\frac{\partial}{\partial t} } { \Phi } = 0$$
\end{block}
\end{frame}

\begin{frame}{\mytitle{Wave function ansatz}}
\begin{block}{Basis $\{\phi_k(\lambda_1,\ldots,\lambda_M)\}_{k=1}^N$: }
$$\myvec{\Psi}=\sum_{i=1}^N C_k\phi_k$$
$$\myvec{\delta\Psi}=\sum_{i=1}^N\left( \delta C_k \phi_k + C_k\sum_{l=1}^M\frac{\partial\phi_k}{\partial\lambda_{kl}}\delta\lambda_{kl} \right)$$
$$\delta C_m^*\sum_{k=1}^N\left\langle\phi_m\left|\hat{H}-i\hbar\frac{\partial}{\partial t}\right|C_k\phi_k\right\rangle=0$$
$$\delta\lambda_{mj}\sum_{k=1}^NC_m^*\left\langle\frac{\partial\phi_m}{\partial\lambda_{mj}}\left|\hat{H}-i\hbar\frac{\partial}{\partial t}\right|C_k\phi_k\right\rangle = 0$$
\end{block}
\end{frame}

\begin{frame}{\mytitle{Equation of motions}}
\begin{block}{}
$$\dot{\vec{C}} = -\frac{i}{\hbar}\mathbbm{S}^{-1}\left(\mathbbm{H}-i\hbar\boldsymbol{\tau}\right)\vec{C}\text{ --- vector of coeficients}$$
$$\dot{\Lambda} = -\frac{i}{\hbar}\mathbbm{X}^{-1}\mathbbm{Y}\text{ --- matrix of parameters}$$
$$\mathbbm{S}_{mk}=\myel{\phi_m}{\phi_k},\ \mathbbm{H}_{mk} = \mymean{\phi_m}{\hat{H}}{\phi_k},\ %
  \boldsymbol{\tau}_{mk} = \sum_{l=1}^M\myelr{\phi_m}{\frac{\partial\phi_k}{\partial\lambda_{kl}}}\dot{\lambda}_{kl}$$
$$\mathbbm{Y}_{m}^j=\sum_{k=1}^N\rho_{mk}\left(\mathbbm{H}_{mk}^{(j0)}-%
	                     (\mathbbm{S}^{(j0)}\mathbbm{S}^{-1}\mathbbm{H})_{mk} \right),\ %
  \mathbbm{X}_{mk}^{jl}=\rho_{mk}\left(\mathbbm{S}_{mk}^{(jl)}-%
					     (\mathbbm{S}^{(j0)}\mathbbm{S}^{-1}\mathbbm{S}^{(0l)})_{mk}\right)$$
$$\rho_{mk}=C_m^*C_k,\ \mathbbm{H}_{mk}^{(j0)}=\mymean{\frac{\partial\phi_m}{\partial\lambda_{mj}}}{\hat{H}}{\phi_k}$$
$$\mathbbm{S}_{mk}^{(j0)}=\myell{\frac{\partial\phi_m}{\partial\lambda_{mj}}}{\phi_k},\ %
  \mathbbm{S}_{mk}^{(0l)}=\myelr{\phi_m}{\frac{\partial\phi_k}{\partial\lambda_{kl}}},\ %
  \mathbbm{S}_{mk}^{(jl)}=\myelr{\frac{\partial\phi_m}{\partial\lambda_{mj}}}{\frac{\partial\phi_k}{\partial\lambda_{kl}}}$$
\end{block}
\end{frame}

\begin{frame}{\mytitle{Basis set}}
\begin{block}{Frozen width gaussian wave packets:}
$$\myvec{g_k}=\myexp{\frac{1}{\hbar}\sum_{\alpha}\left[-\frac{1}{2}\omega r_{\alpha}^2 + \xi_{k\alpha}r_{\alpha} + \eta_{k\alpha} \right]}$$
\begin{equation*}
\begin{cases}
\xi_{k\alpha}=\omega q_{k\alpha} + ip_{k\alpha}\\
\eta_{k\alpha}=\frac{1}{4}\left( \ln\left[\frac{\omega}{\pi}\right] - 2\omega q_{k\alpha}^2\right) - iq_{k\alpha}p_{k\alpha}
\end{cases}
\end{equation*}
$$\myvec{g_k}=\left(\frac{\omega}{\pi}\right)^{\alpha/4}%
	      \myexp{\frac{1}{\hbar}\sum_{\alpha}\left[-\frac{1}{2}\omega(r_{\alpha}-q_{k\alpha})^2 + i\frac{p_{k\alpha}}{m_k}(r_{\alpha}-q_{k\alpha}) \right]}$$
$$\frac{\mymean{g_k}{r_{\alpha}}{g_k}}{\myel{g_k}{g_k}} = q_{k\alpha},\ \frac{\mymean{g_k}{-i\hbar\frac{\partial}{\partial r_{\alpha}}}{g_k}}{\myel{g_k}{g_k}} = p_{k\alpha}$$
\end{block}
\end{frame}

\end{document}
