Time dependant Shr\"{o}dinger equation:
\begin{equation}
i\hbar\frac{\partial |\Phi_{ex}\rangle}{\partial t} = \hat{H}|\Phi_{ex}\rangle
\label{eq:1}
\end{equation}
We will consider the following mean value:
$$W=\frac{ \langle \Phi| \hat{H} - i\hbar\frac{\partial}{\partial t} | \Phi \rangle }{\langle\Phi|\Phi\rangle}$$
which, if calculated on exact solutions $|\Phi_{ex}\rangle$ of (\ref{eq:1}), equals to zero, and its variation: $\delta W = 0$

If $|\Phi_{ex}\rangle$ is an exact solution of (\ref{eq:1}), then the norm conservation condition is satisfied:
$$\frac{\partial \langle\Phi_{ex}|\Phi_{ex}\rangle }{\partial t} = 0$$

But we do not know beforehand about norm of arbitrary function $|\Phi\rangle$.
We can consider mean value of $i\hbar\frac{\partial}{\partial t}$:
$$\langle\omega\rangle = \frac{ \langle \Phi | i\hbar\frac{\partial}{\partial t} | \Phi \rangle }{\langle\Phi|\Phi\rangle}$$
Let us will calculate difference between $\langle\omega\rangle$ and its complex conjugate:
$$\langle\omega\rangle-\langle\omega\rangle^* = \frac{i\hbar( \langle\Phi|\frac{\partial \Phi}{\partial t}\rangle + %
							      \langle\frac{\partial \Phi}{\partial t}|\Phi\rangle )}%
						     {\langle\Phi|\Phi\rangle}$$
$$\frac{i}{\hbar}\left(\langle\omega\rangle^*-\langle\omega\rangle\right) = \frac{\partial}{\partial t}\ln\langle\Phi|\Phi\rangle$$
Now we will solve this equation to obtain time dependance of norm:
$$N(t) = N(0)e^P,\text{ where } P = \frac{i}{\hbar}\int_0^t\left(\langle\omega\rangle^*-\langle\omega\rangle\right)\,dt'$$
We see, that if function has a conserved norm, then $P=0,\ \langle\omega\rangle\in\mathbbm{R}$ and hence $W\in\mathbbm{R}$
But if it is an approximate solution we can not guarantee conservation of norm!

But let us assume, that we can construct function $|\Phi'\rangle$, that differes from $|\Phi\rangle$ by angular multiplier:
$$|\Phi'\rangle = |\Phi\rangle\cdot e^Q,\text{ where } Q = \frac{i}{\hbar}\int_0^t\alpha(t')\,dt',\ \alpha(t')\in\mathbbm{C}$$
We need to find parameter $\alpha(t)$, so that norm $\langle\Phi'|\Phi'\rangle$ is conserved. Once again, we will consider a mean value:
$$\langle\omega'\rangle = \frac{\langle\Phi'|i\hbar\frac{\partial}{\partial t}|\Phi'\rangle}{\langle\Phi'|\Phi'\rangle} = %
			  \frac{\langle\Phi|e^{-Q} i\hbar \cdot \frac{i}{\hbar} \alpha(t) e^Q + %
			   i\hbar\frac{\partial}{\partial t}|\Phi\rangle}{\langle\Phi|\Phi\rangle}=\langle\omega\rangle - \alpha$$
If $\langle\omega'\rangle\in\mathbbm{R}$, then:
$$0 = \langle\omega'\rangle-\langle\omega'\rangle^* = \langle\omega\rangle - \langle\omega\rangle^* - 2i\mathit{Im}( \alpha )$$
$$i\mathit{Im}( \alpha ) = \frac{1}{2}\left( \langle\omega\rangle - \langle\omega\rangle^*\right)$$
$$\alpha = \mathit{Re}( \alpha ) + i\mathit{Im}( \alpha ) = %
	   \mathit{Re}( \alpha ) + \frac{1}{2}\left( \langle\omega\rangle - \langle\omega\rangle^*\right)$$
$$\langle\omega'\rangle = \langle\omega\rangle - \alpha = %
			  \langle\omega\rangle - \frac{1}{2}\left(\langle\omega\rangle - %
								  \langle\omega\rangle^*\right) - \mathit{Re}( \alpha ) = %
			  \frac{1}{2}\left(\langle\omega\rangle + \langle\omega\rangle^*\right) - \mathit{Re}( \alpha )$$
$$|\Phi'\rangle = |\Phi\rangle\cdot e^Q = |\Phi\rangle\cdot e^R\cdot e^{-0.5P},%
					      \text{ where } R = \frac{i}{\hbar}\int_0^t\mathit{Re}(\alpha(t'))\,dt'$$
As $N(t) = N(0)\cdot e^P $, we have:
$$|\Phi'\rangle = |\Phi\rangle\cdot e^R\cdot\left(\frac{N(0)}{N(t)}\right)^{1/2}$$

As we've discussed previously, for exact solution of (\ref{eq:1}) $|\Phi_{ex}\rangle$ mean value $W$ equals to zero.
 Let us consider mean values $W'$, calculated on function $|\Phi'\rangle$:
$$W' = \langle H\rangle - \langle\omega'\rangle = %
		      \langle H\rangle - \frac{1}{2}\left(\langle\omega\rangle+\langle\omega\rangle^*\right)+\mathit{Re}(\alpha)$$
Now we need to understand, what $\alpha$ should be to make $W'$ equal to zero:
$$\mathit{Re}(\alpha) = -\langle H\rangle + \frac{1}{2}\left(\langle\omega\rangle+\langle\omega\rangle^*\right)$$
$$\alpha = -\langle H\rangle + \frac{1}{2}\left(\langle\omega\rangle+\langle\omega\rangle^*\right) +%
			      \frac{1}{2}\left(\langle\omega\rangle-\langle\omega\rangle^*\right)=%
	    \langle\omega\rangle - \langle H\rangle = -W$$
But if $\alpha = -W$, we will obtain:
$$|\Phi'\rangle = |\Phi\rangle\cdot e^{ -\frac{i}{\hbar}\int_0^t W\,dt}$$
$$\langle\Phi'|\Phi'\rangle_t = \langle\Phi|\Phi\rangle_t\cdot e^{ -\frac{i}{\hbar}\int_0^t(W-W^*)\,dt} = %
				\langle\Phi|\Phi\rangle_0\cdot e^P\cdot e^{-P} = \langle\Phi|\Phi\rangle_0$$
So we have built functions $|\Phi'\rangle$, that have conserved norm and lead to zero $W'$. 

Let us consider for simplicity function $|\Psi\rangle$ to be from the family of functions $|\Phi'\rangle$. 
This function has conserved norm, and mean value $W$, calculated on this function, is real.
As the norm of $|\Psi\rangle$ doesn't change with time, we can write the following equation:
$$\delta\langle\Psi|\Psi\rangle = \langle\delta\Psi|\Psi\rangle+\langle\Psi|\delta\Psi\rangle=0$$
We will consider only variations $|\delta\Psi\rangle$, that are orthogonal to $|\Psi\rangle$. Then:
\begin{equation}
\langle\delta\Psi|\Psi\rangle=0,\ \langle\Psi|\delta\Psi\rangle=0
\label{eq:2}
\end{equation}
Now we can write down variation $\delta W$. 
For simplicity, we shall denote $\langle\Psi|\hat{H}-i\hbar\frac{\partial}{\partial t}|\Psi\rangle$ as $A$ 
and $\langle\Psi|\Psi\rangle$ as $B$:
$$W = \frac{A}{B},\ \delta W = \frac{B\cdot\delta A - A\cdot\delta B}{B^2}=\frac{\delta A - W\delta B}{B}$$
$$\delta A - W\delta B = \langle\delta\Psi|\hat{H}-i\hbar\frac{\partial}{\partial t}|\Psi\rangle + %
			 \langle\Psi|\hat{H}-i\hbar\frac{\partial}{\partial t}|\delta\Psi\rangle - 
			 W\langle\delta\Psi|\Psi\rangle - W\langle\Psi|\delta\Psi\rangle = $$
$$ = \langle\delta\Psi|\hat{H}-i\hbar\frac{\partial}{\partial t}|\Psi\rangle + %
     \langle\Psi|\hat{H}-i\hbar\frac{\partial}{\partial t}|\delta\Psi\rangle - W\delta\langle\Psi|\Psi\rangle$$
As $\delta\langle\Psi|\Psi\rangle=0,\ \delta W = 0$, we obtain:
\begin{equation}
\langle\delta\Psi|\hat{H}-i\hbar\frac{\partial}{\partial t}|\Psi\rangle = 0\text{ --- Dirac--Frenkel variational principal}
\label{eq:3}
\end{equation}
$$\langle\Psi|\hat{H}-i\hbar\frac{\partial}{\partial t}|\delta\Psi\rangle = 0$$
We shall consider the second equation:
$$\langle\Psi|\hat{H}-i\hbar\frac{\partial}{\partial t}|\delta\Psi\rangle = %
  \left\langle\left.\left(\hat{H}-i\hbar\frac{\partial}{\partial t}\right)\Psi\right|\delta\Psi\right\rangle-%
  i\hbar\left.\left\langle\frac{\partial\Psi}{\partial t}\right|\delta\Psi\right\rangle-%
  i\hbar\left\langle\Psi\left|\frac{\partial}{\partial t}\delta\Psi\right\rangle\right.=$$
$$=\langle\delta\Psi|\hat{H}-i\hbar\frac{\partial}{\partial t}|\Psi\rangle^* -%
   i\hbar\frac{\partial}{\partial t}\langle\Psi|\delta\Psi\rangle = 0$$
Thus, the second equation is a mere consiquence of Dirac--Frenkel variational principal (\ref{eq:3}) and condition (\ref{eq:2}).

Previously we have discussed the case of orthogonal variation $|\delta\Psi\rangle$. 
Arbitraty variations $|\delta\Psi\rangle$ can be rewritten as sum of $|\Psi\rangle$ and $|\delta_{\perp}\Psi\rangle$:
$$|\delta\Psi\rangle = c_{||}|\Psi\rangle + c_{\perp}|\delta_{\perp}\Psi\rangle$$
Variation of $W$ will have the following look:
$$\delta W = \langle\delta\Psi|\hat{H}-i\hbar\frac{\partial}{\partial t}|\Psi\rangle + %
	     \langle\Psi|\hat{H}-i\hbar\frac{\partial}{\partial t}|\delta\Psi\rangle - 
	     W\langle\delta\Psi|\Psi\rangle - W\langle\Psi|\delta\Psi\rangle = $$
$$ = \langle\delta\Psi|\hat{H}-i\hbar\frac{\partial}{\partial t}-W|\Psi\rangle + %
     \langle\Psi|\hat{H}-i\hbar\frac{\partial}{\partial t}-W|\delta\Psi\rangle = $$
$$ = 2\mathit{Re}(c_{||})\langle\Psi|\hat{H}-i\hbar\frac{\partial}{\partial t}-W|\Psi\rangle - %
     c_{\perp}^*\langle\delta_{\perp}\Psi|\hat{H}-i\hbar\frac{\partial}{\partial t}-W|\Psi\rangle - %
     c_{\perp}\langle\Psi|\hat{H}-i\hbar\frac{\partial}{\partial t}-W|\delta_{\perp}\Psi\rangle $$
The first term equals zero, because:
$$W = \frac{\langle\Psi|\hat{H}-i\hbar\frac{\partial}{\partial t}|\Psi\rangle}{\langle\Psi|\Psi\rangle},\ %
      \langle\Psi|\hat{H}-i\hbar\frac{\partial}{\partial t}-W|\Psi\rangle = 0$$
The last two terms are equal to zero due to Dirac--Frenkel variational principle.


